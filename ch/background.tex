\section{背景}
关于共识的研究最早可以追溯到1959年[]。计算机领域的共识主要研究分布式系统中所有节点达成一致性的问题。具体到区块链中的共识算法而言,其目的是决定出块权(及其奖励)的归属,需要满足下面两个基本性质。(各文献的描述存在细微差别)

\begin{itemize}
\item 一致性(consistency, safety)所有诚实节点最终(对某个提案)达成一致。
\item 活性 (liveness)所有诚实节点发起的交易最终都会被记录,
\end{itemize}
同时,一个好的区块链共识算法有如下基本指标:
\begin{itemize}
\item 去中心化:出块权不应集中在某个个体或团体手里。
\item 抗女巫攻击:鉴于在区块链上建立新账户是没有成本的,共识算法的模型应该能够遏制用户通过大量建立新账户来提升自己获得出块权的概率。PoW和PoS分别用算力以及才产作为竞争出块的依据来防止女巫攻击。
\item 每秒交易次数(TPS):TPS反映出区块链系统的吞吐量。目前比特币的TPS为10,以太坊为15-25。
\item 交易确认时间:交易确认时间影响链上交易的效率以及用户体验。目前比特币的确认时间为1小时,以太坊为3分钟左右。
\end{itemize}

\subsection{思考}
现在普遍认为区块链系统中也有不可能三角的存在,即去中心化,安全性和可扩展性不能同时达到。故设计新共识时需要有所取啥。

同时,女巫攻击以及去中心化(反独裁性)也难以同时保证。假设财产为$a(b)$的账户能产生的收益(由共识算法决定的,如出块奖励)为$f(a)(f(b))$。若能抵抗女巫攻击,理论上需要满足$f(a+b)>f(a)+f(b)$。这样可以推出$f(n)>nf(1)$,意味着大户的绝对统治($n$可以达到上亿级别)。

值得注意的是,上述分析不仅仅适用于PoS机制,即使是对PoW机制$f(a)$也可以理解为财产为$a$的用户的挖矿效用,因为两者存在正相关关系。所以现在的比特币也可以理解为被矿池所统治的中心化系统,但同时浪费了额外的资源进行挖矿。相比而言,PoS机制没有挖矿,但同时也是一个受资本操控的系统。

{\color{red}是否存在矿工证明和系统利益一致的共识?}

例如,矿工通过提升节点性能来挖矿,并且同时能够提升整个系统的TPS,但是这样可能造成安全性降低(见第二章,GHOST)。 或者通过吸引新用户加入来挖矿,但这样除了IOTA这种DAG系统之外,节点数目超出系统容量之后反而会降低系统性能。

其他思考:

是挖矿还是看资产?

是否要有委员会/准入门槛?

是否支持智能合约?

是否考虑分片?或者用DAG?

