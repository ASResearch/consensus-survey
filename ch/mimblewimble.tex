\section{Mimblewimble*}
2016年8月2日一完全匿名的用户在比特币论坛(Bitcoin IRC channel\footnote{https://en.bitcoin.it/wiki/IRC\_channels})上发布了一份名为mimblewimble的文档,随即引起社区成员的广泛关注。2个月之后社区成员Poelstra对原始文章进行了整理并提出了新的见解\cite{poelstra2016mimblewimble}。

该研究主要实现了两个目标:压缩了比特币的存储空间。实现了交易记录中转账金额的隐藏,即,所谓可信交易(CT, confidential transaction)。%下面就这两点进行具体描述。

目前比特币区块链存储的交易信息已达100Gb之多。其主要原因在于用于每一笔交易的UTXO(unspent transaction output,为比特币核心概念之一)都需要被签名,导致存储消耗高。Mimblewimble在不使用可信交易情况下能将这个数字缩减为15Gb。Mimblewimble的核心思想为,每一笔Mimblewimble交易只需要对所谓“多余(EC, Excess)”UTXO进行签名即可。